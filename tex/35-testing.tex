\chapter{Тестирование}
\label{cha:ch_5}
Процесс тестирования необходимо разделить на этапы:
\begin{enumerate}
	\item Низкая нагрузка --- размерность РТДРП в 10-20 элементов;
	\item Стандартная нагрузка --- РТДРП размерностью 30-60;
	\item Максимальная нагрузка --- РТДРП размерностью от 80 до 100.
\end{enumerate}

Раздел тестирования построен следующим образом:
\begin{enumerate}
	\item Указываются входные параметры (размерность РТДРП);
	\item Путь к директории, где расположены svg-файлы (/Tests/$\textbf{размерность РТДРП}$/svg);
	\item Количество svg-файлов в директории;
	\item Время отрисовки всех svg-файлов;
	\item Время выполнения алгоритма Дейкстры.
\end{enumerate}

\begin{table}[]
\centering
\label{my-label}
\begin{tabular}{|l|l|l|l|}
\hline
\textbf{Размерность} & \multicolumn{1}{|p{3cm}|}{\textbf{Кол-во svg-файлов}} & \multicolumn{1}{|p{3cm}|}{\textbf{Время отрисовки (мс)}} & \multicolumn{1}{|p{3cm}|}{\textbf{Время выполнения алгоритма Дейкстры (мс)}} \\ \hline
21                   & 6                          & 117.321                       & 13.011                                            \\ \hline
10                   & 4                          & 47.4077                       & 9.29                                              \\ \hline
50                   & 7                          & 549.783                       & 52.0214                                           \\ \hline
40                   & 8                          & 431.758                       & 73.3365                                           \\ \hline
60                   & 9                          & 1626.04                       & 312.842                                           \\ \hline
100                  & 11                         & 8389.83                       & 5161.01                                           \\ \hline
\end{tabular}
\caption{Результаты тестирования}
\end{table}