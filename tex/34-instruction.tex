\chapter{Описание работы программы}
\label{cha:ch_4}

При запуске файла main.cpp пользователю предлагается ввести несколько значений:
\begin {enumerate}
	\item dimension - размерность сетки (по условию от 10 до 100)
	\item startX, startY - координаты начальной ячейки (не превышают размерности сетки)
	\item finishX, finishY - координаты конечной ячейки (не превышают размерности сетки)
\end {enumerate}
Если данные не соответствуют требованиям выводится сообщение "Invalid input".

Далее создается объект класса HexagonGrid, то есть создается сетка с учетом закрытых ячеек. А затем выполняется алгоритм. Для каждого фронта волны создается svg файл, в котором числами отмечены этот и все предыдущие фронты. В конце алгоритма, если путь найден, создается итоговый файл \textbf{00\_FINAL.svg}, в котором отображен путь. Если в процессе выполнения алгоритма не найдено свободных соседних ячеек, а финиш не достигнут, выводится сообщение "Path is not found".

